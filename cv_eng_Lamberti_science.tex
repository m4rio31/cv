\newcommand*{\skypesocialsymbol} {%
  \protect\raisebox{-0.085em}{%
\protect\begin{tikzpicture}[y=0.08em,x=0.08em,xscale=0.022,yscale=-0.022, inner sep=0pt, outer sep=0pt]
\protect\path[fill=color2,even odd rule] (487.6550,288.9690) .. controls (489.0610,278.5690) and
  (489.8700,267.9960) .. (489.8700,257.2330) .. controls (489.8700,128.0770) and
  (384.5990,23.3610) .. (254.7670,23.3610) .. controls (241.8630,23.3610) and
  (229.2120,24.4210) .. (216.9010,26.4410) .. controls (194.8280,12.0570) and
  (168.5590,3.6740) .. (140.2880,3.6740) .. controls (62.7660,3.6740) and
  (0.0000,66.4820) .. (0.0000,143.9800) .. controls (0.0000,172.1780) and
  (8.2990,198.3740) .. (22.5900,220.3690) .. controls (20.6650,232.3860) and
  (19.6810,244.6920) .. (19.6810,257.2290) .. controls (19.6810,386.4050) and
  (124.8980,491.1100) .. (254.7660,491.1100) .. controls (269.4230,491.1100) and
  (283.6930,489.6840) .. (297.5620,487.1780) .. controls (319.1120,500.5470) and
  (344.4960,508.3260) .. (371.7080,508.3260) .. controls (449.2100,508.3260) and
  (512.0010,445.5020) .. (512.0010,368.0120) .. controls (511.9980,338.7190) and
  (503.0410,311.4840) .. (487.6550,288.9690) -- cycle(276.7400,429.5960) ..
  controls (202.0340,433.4870) and (167.0750,416.9590) .. (135.0500,386.9050) ..
  controls (99.2850,353.3370) and (113.6520,315.0500) .. (142.7900,313.1040) ..
  controls (171.9120,311.1590) and (189.3980,346.1160) .. (204.9410,355.8400) ..
  controls (220.4650,365.5280) and (279.5340,387.6000) .. (310.7350,351.9320) ..
  controls (344.7100,313.1040) and (288.1410,293.0120) .. (246.6760,286.9300) ..
  controls (187.4730,278.1640) and (112.7260,246.1370) .. (118.5410,183.0230) ..
  controls (124.3580,119.9490) and (172.1230,87.6090) .. (222.3910,83.0470) ..
  controls (286.4680,77.2300) and (328.1820,92.7540) .. (361.1760,120.9070) ..
  controls (399.3270,153.4360) and (378.6840,189.8010) .. (354.3770,192.7270) ..
  controls (330.1660,195.6360) and (302.9730,139.2230) .. (249.5860,138.3750) ..
  controls (194.5590,137.5110) and (157.3690,195.6360) .. (225.3000,212.1590) ..
  controls (293.2660,228.6640) and (366.0500,235.4450) .. (392.2610,297.5760) ..
  controls (418.4900,359.7130) and (351.5070,425.7010) .. (276.7400,429.5960) --
  cycle;
\protect\end{tikzpicture}}%
  ~}

\documentclass[11pt,a4paper,roman]{moderncv}        % possible options include font size ('10pt', '11pt' and '12pt'), paper size ('a4paper', 'letterpaper', 'a5paper', 'legalpaper', 'executivepaper' and 'landscape') and font family ('sans' and 'roman')


% moderncv themes
\moderncvstyle{classic}                             % style options are 'casual' (default), 'classic', 'oldstyle' and 'banking'
\moderncvcolor{green}                               % color options 'blue' (default), 'orange', 'green', 'red', 'purple', 'grey' and 'black'
%\renewcommand{\familydefault}{\sfdefault}         % to set the default font; use '\sfdefault' for the default sans serif font, '\rmdefault' for the default roman one, or any tex font name
\nopagenumbers{}                                  % uncomment to suppress automatic page numbering for CVs longer than one page

% character encoding
\usepackage[utf8]{inputenc}                       % if you are not using xelatex ou lualatex, replace by the encoding you are using
%\usepackage{CJKutf8}                              % if you need to use CJK to typeset your resume in Chinese, Japanese or Korean

% adjust the page margins
\usepackage[scale=0.77]{geometry}
%\setlength{\hintscolumnwidth}{3cm}                % if you want to change the width of the column with the dates
%\setlength{\makecvtitlenamewidth}{10cm}           % for the 'classic' style, if you want to force the width allocated to your name and avoid line breaks. be careful though, the length is normally calculated to avoid any overlap with your personal info; use this at your own typographical risks...

% personal data
\name{Mario}{Lamberti}
\title{Curriculum Vit\ae}                               % optional, remove / comment the line if not wanted
%\address{+39 340 42 61 266}{mario.lamberti31@gmail.com}{Anno di nascita: 31 Gennaio 1993}% optional, remove / comment the line if not wanted; the "postcode city" and and "country" arguments can be omitted or provided empty
\phone[mobile]{+39~340~4161266}                   % optional, , remove / comment the line if not wanted
\email{mario.lamberti31@gmail.com}                               %  remove / comment the line if not wanted

\extrainfo{\skypesocialsymbol mario.lamberti31 \\
			Date of birth: 31 January 1993 \\ 
			Place of birth: Salerno (SA), Italy}


\begin{document}
\makecvtitle
\section{Education}
\cventry{2017--2020}{Master of Science in Physics (LM-17)}{Università degli Studi di Milano}{Italy}{}
{Relevant topics: Electroweak Interaction, Strong Interaction, Standard Model and Higgs sector, Quantum Field Theory and Renormalization, Particle Detectors. \\
Thesis title: "Measurement of differential cross sections for Higgs boson production in the $\gamma\gamma$ decay channel at $\sqrt{s} = 13$ TeV with the ATLAS experiment." \\
Supervisors: Prof. Marcello Fanti, Dott. Ruggero Turra \\
110/ 110 \\
CERN Document Server link: \url{https://cds.cern.ch/record/2746556?ln=en} \\
INFN Document Server link: \url{http://www.infn.it/thesis/thesis_dettaglio.php?tid=529318}}
\cventry{2018--2019}{Erasmus\scriptsize{+} \normalsize{Exchange Programme}}{Albert-Ludwigs-Universität Freiburg}{Germany}{}
{Relevant topics: Advanced Particle Physics, Particle Detectors, Advanced Quantum Field Theory. \\
Period of stay:  October 2018 - March 2019}
\cventry{2012--2017}{Bachelor of Science in Physics (L-30)}{Università degli Studi di Milano}{Italy}{}
{Relevant topics: Classical Physics, Quantum Physics, Electromagnetism, Linear Algebra, Mathematical Analysis. \\ 
Thesis title: "Interferometria quantistica con elettroni e positroni: studio preliminare per esperimento QUPLAS." \\
Supervisors: Prof. Fabrizio Castelli, Prof. Marco Giammarchi, Dott. Simone Sala \\
94/ 110}
\cventry{2006--2011}{Diploma}{Liceo Classico G.D. Romagnosi}{Parma (PR)}{Italy}
{Maturità classica}
\section{Working Experiences}
\cventry{2021--present}{Big Data Engineer at DataReply}{Milan}{Italy}{}
{I am specializing in the Google Cloud Platform services for creating and manipulating databases and implementing machine learning models, in order to automate the productive processes related to data. \\
DataReply Website: \url{https://www.reply.com/data-reply/it/}}
\section{Research activities}
\cventry{2020--present}{Associate Member of the ATLAS Collaboration at CERN}{Geneva}{Switzerland}{}
{I am working in the H $\rightarrow$ $\gamma\gamma$ cross-sections working group within the ATLAS Collaboration. My contributions to the research are focused on the determination of the Higgs boson fiducial differential cross sections using a different statistical approach, in order to cross-check and validate the values extracted with the official statistical framework.
This work is part of an analysis aimed at publishing an article by the ATLAS Collaboration.}
\cventry{2020}{Measurement of differential cross sections for Higgs boson production in the $\gamma\gamma$ decay channel at $\sqrt{s} = 13$ TeV with the ATLAS experiment \small{(Master's degree)}}{}{}{}
{The thesis project covered the measurement of fiducial differential cross sections for the Higgs boson production for a set of selected observables, in the H $\rightarrow$  $\gamma\gamma$ decay channel. This channel has the advantage of having a fully reconstructed final state and a very good invariant mass resolution of the Higgs peak. Within the analysis I have also tested two different unfolding methods and developed for both of them a statistical model, useful in the extraction of the differential cross section values. The analysis presented in this work is based on the LHC Full Run2 (L = 139 fb$^{-1}$ ), with some variations and additional contributions with respect to the previous papers published by the ATLAS Collaboration. \\
CERN Document Server link: \url{https://cds.cern.ch/record/2746556?ln=en}}
\cventry{2017}{Quantum interferometry with electron and positron: a preliminary analysis for the QUPLAS experiment \small{(Bachelor's degree)}}{}{}{}
{The main purpose of the QUPLAS program is to investigate the interferometric and gravitational properties of the antimatter and of the matter-antimatter complex systems. My thesis contribution has consisted in a preliminary study for the interferometer experimental setup within the QUPLAS Collaboration. This work is focused on the optimization of an electron beam produced by an electron gun, useful for carrying out alignment and calibration tests on the interferometric system of the QUPLAS 0 design phase. We proceeded using programming tools in python language and the use of a special SimIon software to build a computational simulation that would reproduce the previously acquired experimental data.}

\section{Papers}
\cventry{2021}{Measurements and interpretations of Higgs-boson fiducial cross sections in the diphoton decay channel using 139 fb$^{-1}$ of $pp$ collision data at $\sqrt{s} = 13$ TeV with the ATLAS detector [SHORTLY BEING PUBLISHED]}{}{}{}{}
\section{Certifications}
\cventry{2020}{Data Analysis with Python}{Powered by IBM}{}{}{}
\cventry{2020}{Python for Data Science}{Powered by IBM}{}{}{}

\section{Language skills}
\cvitemwithcomment{Italian}{Mother tongue}{}
\cvitemwithcomment{English}{B2}{}
\cvitemwithcomment{German}{A2}{}

\section{Programming skills}
\cvitem{Languages \small{(advanced)}}{Python, Python3, \LaTeX}
\cvitem{Languages \small{(intermediate)}}{C++, Bash, XML, SQL}
\cvitem{Languages \small{(basic)}}{Mathematica, JavaScript, HTML4/5, CSS}
\cvitem{Tools \& frameworks}{ROOT, RDataFrame, PyROOT, RooFit, HTCondor, Jupyter Notebook, Python VirtualEnv, Numpy, Pandas, Matplotlib, Docker, MySQL, Git, Google Cloud Platform}

\section{About Me}
My research interests are focused the most in the field of particle physics and in the comprehension of the Standard Model of Particle Physics (SM) which it is based on. Among the wide range of topics in it, I find of great interest the Higgs sector and the Higgs-related processes. One of the main hot-spot in the present particle physics research so far, this particle represents one of the building blocks of the Standard Model and it is responsible for some of the most fundamental processes of the latter. My approach to the study of physics has always been mostly experimental with a strong theoretical background, leading to a good knowledge of the Standard Model-related processes and of the Quantum Field Theory (QFT) used for their description. Alongside, I became very interested in programming, which got me gradually into the knowledge of several programming languages, making a daily use of them during the research periods. Other scientific interests comprehend astroparticles and general relativity. During my work experience I'm having the opportunity to acquire more skills in the management of a BigData project, improving as well my knowledge of machine learning processes using the Google Cloud Platform framework and its resources. Besides physics and science, I have acquired skills useful both during the everyday life and in a scientific research program. I have acquired communication skills during a multi-annual experience in private lessons of physics and math for high school students; teamwork skills during the scouting experience, in particular as a tutor for the activities; relationship skills in a multi-cultural environment during the Erasmus+ exchange programme in Freiburg, Germany.

\end{document}
